\documentclass{article}
\usepackage[utf8]{inputenc}
\usepackage{romannum}
\usepackage{amsfonts}
\usepackage{amssymb}
\usepackage{fancyhdr}
\usepackage{graphicx}
\usepackage{t1enc}
\usepackage{pdfpages}
\usepackage[magyar]{babel}
\usepackage[utf8]{inputenc}
\usepackage{amsmath}
\usepackage{mathtools}
\usepackage{pdfpages}
\usepackage{ marvosym } 
\usepackage{wrapfig}
\usepackage{hyperref}
\usepackage{pgfplots}

\pgfplotsset{width=\textwidth,compat=newest}
\hypersetup{
    colorlinks,
    citecolor=black,
    filecolor=black,
    linkcolor=black,
    urlcolor=black
}
\usepackage{romannum}
\usepackage{amsfonts}
\usepackage{amssymb}
\usepackage{fancyhdr}
\usepackage{graphicx}
\usepackage{t1enc}
\usepackage{svg}
\usepackage[magyar]{babel}
\usepackage[left=2cm,right=2cm,top=2cm,bottom=2cm]{geometry}

\title{K+F Projekt dokumentáció}
\author{}
\date{2}
\pagestyle{fancy}
\lhead{K+F Projekt dokumentáció}
\rhead{ACSG Kft.}
\cfoot{\thepage. oldal}

\begin{document}
\pagenumbering{arabic}

\begin{center}
    \hrule
    \vspace{0.5cm}
    \begin{Huge}
    K+F projekt dokumentáció\\
    ACSG Kft.\\
    \end{Huge}
    \vspace{0.5cm}
    \begin{huge}
    Időszak:\\
    \end{huge}
    \begin{Large}
    2023.01.01. - 2023.02.28.\\
    \end{Large} \vspace{10pt}
    \begin{large}
    Készítette: Wenesz Dominik\\
    \end{large}\vspace{0.5cm}
    \hrule
    
  \end{center}
  \begin{figure}[b]
    \centering
    \includegraphics[]{acsg.png}
\end{figure}
  


\thispagestyle{empty}
\setcounter{page}{0}




\newpage
\tableofcontents
\newpage

\section{Összefoglaló}
A dokumentum által tekintett időintervallumon belül megtörtént a robotkar (SCARA)
pontosságának tesztelése a munkatérben, melynek eredményeként elmondható, hogy 
a csatlakozóházak megfelelően precíz manipulálásához megfelelő az alkalmazott robotkar.\\[5pt]
A pontosság mérésének tükrében a későbbi digitális iker tervezése is megkezdődött, de 
a tekintett időszakban mindösszesen a szimulációs keretrendszer beállítása indult el 
egy Linux operációs rendszerű PC-n.\\[5pt]
A mélytanulásos objektumdetektálás tanítóadat generáló szoftvere a jelenlegi teszt 
alkalmazásokhoz teljesen elkészült, a későbbiekben a rendszerszintű integráció 
kapcsán lesz csak szükséges további változtatás eszközölése rajta.\\[5pt]
Folytattuk a különböző Deep Learningen alapuló objektum detektáló hálóarchitektúra 
vizsgálatát a csatlakozóházak detektálására.\\[5pt]
Az objektum detektáláson túl a pozíciómeghatározás lehetséges megoldásait vizsgáltuk,
melynek eredményei lentebb láthatók, illetve az objektum detektálást részletesen tárgyaló
dokumentumba felvezetésre kerültek.


\section{Objektum detektáló alrendszer}
Az előző dokumentumban kifejtésre került, hogy a deep learning egy jó alternatívát jelent a hagyományos képfeldolgozási 
módszerekre többek közt az objektumdetektálás területén belül is. 

\subsection{Vizsgált hálóarchitektúrák}
Elsősorban a különböző YOLO architektúrákat vizsgáltuk először valós adatokon (kisebb adathalmaz), 
majd szintetikus képeket, illetve vegyes adathalmazon is.\\
Jó eredmények születtek, ezek részletezése az összefoglaló dokumentációban 
tekinthető meg, illetve a tezstelési jegyzőkönyvekben.\\[5pt]
Megjegyzésként pár képpel demonstrálva a haladást:
\begin{figure}[h]
  \centering
  \includegraphics[scale=0.5]{rossz.png}
  \caption{Nem megfelelően generált képek esetén tanítás után}
\end{figure}
\begin{figure}[h]
  \centering
  \includegraphics[scale=0.7]{jo.png}
  \caption{Megfelelően generált képek esetén tanítás után}
\end{figure}
\section{Tanítóadat generálás}
\subsection{Probléma ismertetése}
A deep learning alkalmazhatósága szinte bármilyen tudományterületen 
megkérdőjelezhetetlen, struktúrált és struktúsálatlan adatra is egyaránt 
használható. Jellemzően két robléma merül fel ha a mélytanulásról beszélünk, 
az egyik a nagyságrendekkel nagyobb tanítóadatot igény, illetve a tanításhoz 
szükséges számítási kapacitás és idő.\\[5pt]
Könnyedén belátható, hogy egy felxibilis objektumdetektáló rendszerhez 
nagyszámú képre van szükség, adott esetben gyakran, ez ha emberi erőből végezzük 
időben és kiadásokban is nagy igényekkel rendelkezik.
\subsection{Megoldási lehetőségek}
Egy megoldási lehetőség a fentebb említett adatigényre, ha szintetikusan 
generáljuk azokat. Természetesen ez már önmagában hibákat szülhet, hiszen 
a generált képekkel a valóságot lefedni igen nehéz feladat tud lenni.
A képgenerálás mellett az adatdúsítás is igen hasznos eszköz ebben a problémakörben,
ez csupán annyit jelent, hogy képfeldolgzási technikákkal manipuláljuk a tanítóadatokat 
(zaj bevezetése, elforgatás stb..) és így sokszor nagyobb adatbázishoz jutunk.\\[5pt]
Az adatdúsítás viszonylag egyszerű feladat, bevett módszerek vannak rá, azonban 
a szintetikus képgenerálás (azaz számítógéppel történő szimuláció gyakorlatilag) már 
korántsem ilyen triviális.\\[5pt]
Egy további tényező mely nehezítést von be az, hogy a képgeneráásnak viszonylag gyorsan is kell 
történnie (legalább olyan gyorsan, mintha valós képeket készítenénk), 
azaz egy jó választás lenne valamilyen game-engine használata például a Unity-é, 
hiszen ezek nagyon gyors renderelésre lettek kifejlesztve. Azonban 
nagyon fontos tényező a valóság reprezentálása, lefedése és tipikusan a
számítógépes játékokat meghajtó motorok csak addig generálják a képeket, amíg 
kellőképpen valósnak, szépnek látszik az emberi szem számára, ez azonban 
még nem a valós képhez haosnlító eredményt jelent, több fizikai tényező is 
elhanyagolásra kerül.\\[5pt]
Az általunk használt megoldás egy fizikai renderelésen alapuló nyílt forráskódú 
szoftver és egyben könyvtár a Blender használatára épül. Tetszőlegesen 
komplex kép generálható benne, mely azon fizikai paramétereket is figyelembe veszi, 
amiket a fentebb említett gyorsabb környezetek nem. Emellett a sebessége is elfogadható, 
GPU-kra optimalizált, így egyszerre több szálon párhuzamosan tud futni a 
renderelés.\\[5pt]
Eze szoftverre építettük még egy nyílt forráskódú segédkönyvtár bevonásával 
a képgeneráló programunkat, melynek egy CAD file és bemeneti paraméterek (hány képet generáljunk stb..) 
kell csupán és tetszőleges tanító adatot tudunk létrehozni vele (jellemzően egy képhez 5-30 másodperc szükséges).\\[5pt]
Fontos még megemlíteni, hogy a hálókhoz nem elég csupán a nyers képek megléte, 
azokat címkézni is kell, azaz másik fájlokban megadni, hogy egy-egy adott képen 
milyen osztályú objektum és hol helyezkedik el, ez humán munkával szintén 
igen költséges lenne, de a program ezt is megoldja. További részletesebb leírás és eredmények 
az objektumm detektálás dokumentációban.
\begin{figure}[h]
  \centering
  \includegraphics*[scale=0.5]{data.png}
  \caption{Generált adat}
\end{figure}
\subsection{Vegyes tanító adat}
A szintetikus adatok mellett ha valósakat is tartalmaz az adatbázisunk, 
az legtöbbször a neurális háló eredményességében javulást eredményez.
\subsection{Valós adatra illesztés}
Egy-egy adott hálót a szintetikus vagy vegyes adatokon tanítás után 
finomhangolni lehet ha megfelelő minőségű valós képeket tanítjuk tovább 
jóval kevesebb ideig.
\subsection{További fejlesztési irányok}
Egy másik igen érdekes fejlesztési irány a képeknek neurális hálókkal 
történő generálása.




\section{Pozíciómeghatározás}
A fentebb említett hálóarchitektúrák önmagukban mindösszesen az objektumdetektálásra képesek, mint azt az előző
részekben megállapítottuk, jelen alkalmazási formára a sebesség igényénél fogva elegendő a bounding-box típusú
objektu felismerést alkalmaznunk, hiszen elegendő adatot tartalmaz a robotvezérléshes és szignifikánsan gyorsabb,
mint a szegmentálással is kiegészített hálóarchitektúra.\\[5pt]
Azonban a megfelelő fészekbe való pozícionáláshoz további input paraméterekre van szükségünk, az objektum elfordultsága
is nélkülözhetetlen paraméter ha egy kamerával szeretnénk megoldani a feladatot. Azonban itt több lehetőségünk is van a 
megvalósításra, a dokumentum által tárgyalt időszakban ezen lehetőségeket térképeztük fel és alkalmazhatóságukat 
az adott problémán elkezdtük megvizsgálni.\\
Az általunk megfontolandónak talált megközelítések a következők:
\begin{itemize}
  \item Egy a munkaterület felett elhelyezett kamerától különálló kamera, mely a fészek felett van elhelyezve
  és hagyományos képfeldolgozási technikával (pl.: szögkülönbség meghatározás) megfelelő rotációs értéket adja a 
  robotvezérlőnek. Természetesen léteznek nem explicit szögmeghatározó algoritmusok is, melyek egy adott hibaminimalizációra,
  vagy mintakövetésre törekszenek, azaz kvázi regresszív módszerrel közelítik a megfelelő orientációt, mellyel a fészekbe
  kell kerülnie a csatlakozóháznak.\\
  Ez a megoldás viszonylag egyszerű, azonban mivel két kamera szükséges hozzá (munkatér nagysága miatt is), illetve 
  nem minden csatlakozó esetén feltétlenül hibamentesen generalizálható, ezért alkalmazhatóságát meg kell vizsgálni részletesebben.
  \item Az előző esetet egy kamerával is megoldhatjuk ha azt a robotkar és a munkatér, illetve a fészek geometriája
  lehetővé teszi.
  \item A hagyományos képfeldolgozási módszereken túl ebben az esetben is használhatunk neurális hálózatokat, azaz
  deep learingen alapuló módszereket.\\
  Egyik módszer, hogy már az objektumdetektáló neurális hálózatot bővítjük ki egy orientációt meghatározó résszel. 
  Ekkor annyi szabadságunk van, hogy tetszőleges kondíciókat adhatunk meg, azaz lehetséges a fészekbe való behelyezés
  több variációval való kiegészítése, illetőleg regressziós módon megadhatunk 1-3 elfordulási komponenst (tengelyek körül),
  de tekintve, hogy SCARA robottal dolgozunk, így elegendő egy elfordulással operálni ha már megállapítottuk az 
  objektumról, hogy felvehető.\\
  A rendszer ilyen szintű bővítésével eljutunk oda, hogy implicit meghatározzuk az objektum 3D térben vett 6 szabadságfokú
  pozícióját, mely a mélytanulás ilyen típusú alkalmazásában egy teljesen külön álló kutatási terület, melyre 
  léteznek megoldási módok, melyeket a későbbiekben tesztelni is fogunk, azonban jelen alkalmazásnál SCARA robotnál 
  egyszerűbb modellel is képesek vagyunk elérni a mefelelő pozícionálást elméleti szinten.
  \item A pozíció meghatározása nem feltétlen kell, hogy explicit módon történjen, ha egy adott objektum osztályon 
  belül úgynevezett keypoint-okat keresünk, akkor azok elhelyezkedéséből determinálható az objektum pozíciója.
  Ennek előnye, hogy rengeteg információt tudunk meghatározni egy lépésben, de jellemzően egy ilyen háló betanítása
  több tanítóadatot igényel és nagyobb osztályszám esetén nem feltétlenül konvergál a valósághoz, továbbá természetesen
  a végrehajtási idő is jeentősen megnő a háló komplexitásával együtt.
\end{itemize}
\subsubsection{Megvizsgált módszerek}
Az adott időintervallumban egyelőre főként csak a lehetséges megoldások elméleti kidolgozása zajlott, illetve
a hagyományos képfeldolgozáson alapuló módszerek közül elkezdtük az algoritmusok implementációját is, persze
az OpenCV-ben találhatók hasonló algoritmusok, de a feladat specifikussága miatt mindenképp akár ezeket felhasználva is, 
de saját algoritmust fejlesztenünk a robosztusabb megoldáshoz.
\section{Robot manipulátor}
Miután már kiválasztásra került a technológiai szempontból optimális végberendezés, annak valódi környezetben 
való tesztelését végeztük, továbbá magának a robotvezérlőnek a beállítása, kalibrálása, a robot felprogramozása és 
egyszerűbb illetve összetettebb mozgások megvalósítása LUA programnyelven.\\[5pt]
Többek között a maximális csuklógyorsulást és csuklósebességet teszteltük. Ezek a robot adatlapjában is meg vannak adva 
számszerűen, azonban mindig érdemes az ilyen jellegű validálás, főként mivel az adatlapok az ideális, adott modellezési 
módhoz tartozó értékeket tartalmazzák.\\
(...)

\section{Irodalomjegyzék}







\end{document}